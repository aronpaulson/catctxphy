%\nocite{0b855cc5}
%\nocite{476fe2a3}
Let us first discuss the {\glqq}built-in continuity{\grqq}. As humans we claim that we experience nature as a smooth continuum (think of this intuitively for the moment and not formally). As it proved very useful over the past centuries to describe nature mathematically we hence need a mathematical idea of what a smooth continuum is. In Euclidean geometry continuity is built-in. It is a primitive concept. One does not explain what the continuity of a line is but one is expected to know from experience what it is. Now Euclidean style geometry is a bit limited as a theory to describe modern physics. So, historically, Euclidean geometry was gradually replaced by mathematicians around 1900 with the nowadays mostly accepted mathematical theory of everything: some version of set theory. Set theory is all about collections of things and one has to model everything one wants to describe in terms of collections of things. In particular, a smooth continuum has to be modeled like that. Most mathematicians agreed (more or less forced) to use Zermelo-Fraenkel set theory with choice (abbr. ZFC) and reasoning with classical logic. One day someone came up with the idea of the somehow magic set of real numbers $\mathbb{R}$ as a model for a continuum in ZFC. One can show that in ZFC with classical logic one cannot answer the question if there is something between the natural numbers $\mathbb{N}$ and the real numbers $\mathbb{R}$ in size. So the real numbers could be way bigger than one might think when looking at the idea of real numbers. The idea is to approximate $\mathbb{R}$ by the (not so magic - if one believes in infinity) set of rational numbers $\mathbb{Q}$ in ZFC.\footnote{this is the very idea of analysis: approximate a smooth continuum sufficiently well} However, there are at least two reasonable ways to do this: Dedekind cuts and the Cantor construction. To prove these constructions to be the same one has to assume a countable choice axiom which means that we can magically choose one element from each set of a countably infinite collection of sets. But we do not know which elements we choose and hence we do not really know \underline{why} the Dedekind cuts and the Cantor construction yield the same real numbers. Some mathematicians\footnote{including the authors} consider this philosophically unsatisfactory. Of course, this seems not so bad here since one is actually only interested in the fact that Dedekind cuts and the Cantor construction are the same and not directly why they are. But there is something more severe following from the choice axiom of ZFC: the Banach-Tarski paradox. This means that we can decompose a unit ball in $\mathbb{R}^{3}$ into five disjoint sets and then compose these parts without any {\glqq}deformation{\grqq} into \underline{two} disjoint unit balls in $\mathbb{R}^{3}$. This seems a bit odd under the interpretation of smooth continuum. Anyways, the fact is, that at the moment scientists do not have a theory of everything to describe nature. Even worse, some physical theories which in general yield quite good results struggle with the continuum as is the title and subject of the really worthwhile and easy to read notes \cite{476fe2a3}. The reason why we do not have a physical theory of everything might be that we havn't found the correct set of ZFC to precisely model the physical universe yet. Most mathematical physicist conduct research in this direction. But there are other ideas why we don't have a physical theory of everything yet. Namely the smooth continuum. On the one hand it might be possible that the physical universe is not a smooth continuum at all but rather something discrete as (finitely many) natural numbers. While a version of this is pursued in \cite{0b855cc5} and is certainly within the bounds of possibility we argue in another direction. It might be possible that $\mathbb{R}$ is not the right version of smooth continuum as we pointed out above. And category theory provides an alternative in the guise of topos theory\footnote{category theory subjected to some more axioms as we will see} allowing to formally work with infinitesimals as physicist are used to do informally anyhow. On the other hand there is also so-called synthetic differential geometry which seems to be a worthwhile replacement of ordinary differential geometry. But we refrain from developing all this in these notes. However, \cite{wiki-nlab0000} lists some literature on the subject. As the reader will hopefully learn by reading these notes, category theory can intuitively be understood as a theory of directed paths, that is, paths which can only be gone in one direction. These directed paths have a built-in continuity. Moreover if one demands the directed paths to be reversible one gets a synthetic theory of undirected paths, that is, paths which can be gone in both directions. This is nothing but the study of so-called homotopy $1$-types with the advantage that continuity is built-in. More generally, one dreams of a synthetic directed version of homotopy theory as a formal mathematical theory: higher category theory with $\infty$-categories\footnote{a kind of a space structured by directed paths of any dimension} as only basic mathematical objects. Nobody knows yet if this is a pipe dream. What is known is that an attempt in the undirected case is almost successful. What we are talking about is the Univalent Foundation Program with its Homotopy Type Theory as described in the HoTT book \cite{1ba1603e}. We abbreviate this by UFP-HoTT when referring to it. UFP-HoTT is a synthetic homotopy theory with a built-in continuity with $\infty$-groupoids\footnote{a kind of a space structured by undirected paths of any dimension} as basic mathematical objects. More astonishingly UFP-HoTT can be taken as a constructive foundation for all of mathematics. Constructive in the sense that if something exists one has to say what it is. Even better for the traditionalists among us one can additionally assume a version of the choice axiom which allows to do mathematics as the common mathematician is used to. Anyways, one would wish that some day there will be a directed version of UFP-HoTT. Although we do mostly develop basic category theory modeled in a familiar set theory with classical logic\footnote{but actually anything what we do also works constructively in some sense} for reasons we explain in chapter \ref{chap:cattg} we will often hint at higher category theory and the perks of UFP-HoTT. Conclusively, for this paragraph, let us say that a proper notion of continuum is important both to the foundations of mathematics and the foundation of physics (if the physical universe is a continuum). Hence the continuity aspect is one reason why category theory matters in both subjects.
