%\nocite{797789bc}
%\nocite{a565d200}
%\nocite{0349e8ea}
%\nocite{78202e13}
This will be a rather short section. We just talk about some literature we think is needed to understand \cite{a565d200}. First we have some traditional stuff which most readers who have read so far will already know well enough.
\begin{enumerate}
\item[$\bullet$]
Differential geometry in the scope of Isham \cite{797789bc} (i.p. to understand connections more formally).
\item[$\bullet$]
Algebraic topology in the scope of Hatcher \cite{8b5861fc} and maybe the fiber/cofiber sequence part of May \cite{78202e13} (for a better feeling on homotopy and maybe cohomology).
\item[$\bullet$]
Sheaf and topos theory in the scope of Mac Lane and Moerdijk \cite{c55c71e8}.
\end{enumerate}
Then one needs to understand this modern version of homotopy theory. For that purpose it seems reasonable to have a look at the following.
\begin{enumerate}
\item[$\bullet$]
Synthetic homotopy theory as homotopy type theory such as UFP-HoTT described in \cite{1ba1603e} with an eye on the foundations of mathematics\footnote{we do not know a really good single source for this}.
\item[$\bullet$]
$(\infty,1)$-categories and $(\infty,1)$-topoi in the scope of Lurie \cite{0349e8ea}.
\end{enumerate}
It may well be that the reading list is  not perfect and should be improved by someone who really read and understood \cite{a565d200}. So don't blame us for the potentially flawed and certainly to some extent incomplete list.
\\\\
As last word on these notes we should resolve if the titles initial and terminal context are serious. As you might have realized we like footnotes and to not directly spoil our result we write the answer in a footnote.\footnote{well, if the titles were serious then what we did in these notes would (up to coherent equivalence) be the only way ({\glqq}directed path{\grqq}) to go from chapter \ref{chap:initcontext} to chapter \ref{chap:termcontext} and we claim it is impossible to seriously claim this - hence regard it a pun}
