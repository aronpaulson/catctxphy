This is the main - and by far the longest - chapter of these notes. It is all about category theory (as mathematical theory) with little attention on physics. However we will develop it in a potential mathematical theory of everthing: Tarski-Grothendieck set theory. TG is (almost) ordinary mathematics as you might know it.\footnote{see appendix \ref{chap:tg} for this} Moreover we pay special attention to homotopy theory and higher category theory albeit on a way more informal level than we do for ordinary category theory. Homotopy theory and higher category theory will play a role in almost any section of this chapter. But let us emphasize that it should be possible to understand a good part of this chapter without any prior knowledge of these theories (i.p. homotopy theory). Actually, it should suffice to know some basic set theoretical constructions (unions, instersections, relations, functions, \ldots), ordered sets (pre, partially, totally, well) and a bit about equivalence relations.
\\\\
Let us now talk a bit about the sections of this chapter. Of course, we present the most basic ideas of category theory as you would find them in any text about general category theory but with certain emphases which are not so standard. Let us present this as a list:
\begin{enumerate}
\item[$\bullet$]
The trinity in section \ref{sec:trinity}: categories, functors, natural transformations. What is perhaps special in our notes is the early focus on higher categories which we already introduce in the trinity section \ref{sec:trinity}. We will then later see that this trinity is not so {\glqq}holy{\grqq} as one might guess from from a first glance at category theory but rather an expression of the fact that $0$ and $1$ and $2$ are three natural numbers. 
\item[$\bullet$]
Duality in section \ref{sec:duality}. This is actually addressed in any serious text about category theory since a lack of understanding this can make things a little bloated. Therefore we tried to emphasize this thorughout these notes as much as possible.
\item[$\bullet$]
Constructions on a category in section \ref{sec:constoncat}. We only discussed the ones important to us: product categories and comma categories. There are quite a few more interesting ones we do not discuss such as graphs. For this section there are certainly more complete sources.
\item[$\bullet$]
Universality in section \ref{sec:uni}. This is the most important thing about category theory which is why we extensively discuss it. We strived to be as intuitive as possible (what we actually try throughout the whole notes) with a special interest on parts which are particularly important to physicist (but for mathematicians as well). We start from what {\glqq}universality{\grqq} intuitively means followed by the far reaching Yoneda lemma waiting there for us. It is not exaggerated to say that the Yoneda lemma dominates category in some sense. You will hopefully comprehend what we mean after a perusal of section \ref{sec:uni}. With limits and adjoints we will get to know the arguably most common {\glqq}universal constructions{\grqq}. We will prove most of the standard theorems for those. Then a bit hidden in a subsubsection with Kan extension we get to know a further universal construction which is also of major importance. Especially in homotopy theory they play a role.\footnote{we will later see this more explicity when considering simplicial sets in section \ref{sec:sset}} So be warned and do not underrate the subsubsection \ref{sec:coyoneda2}.
\end{enumerate}
Well, this is what we consider the basics of category theory.
\\
After this basics it is time to look a bit at the big picture of category theory from a modern\footnote{the authors are not full-fledged researchers - so do not expect too much} perspective. This is done in one section with a somewhat obscure title:
\begin{enumerate}
\item[$\bullet$]
Meta-ideas in section \ref{sec:metaidea}. Category theory and especially higher category theory is surrounded by many more or less informal ideas (therefore meta-ideas). We try to present these in a more or less informal style depending on how sophisticated the idea is and how much effort is acceptable for our purposes. The section is full of examples some of which are very important to us while others might be only for illustrative purposes. But almost anything we discuss in this section has more or less direct applications in physics. In particular, we define principal bundles and discuss them the more general setting of (higher) topoi.
\end{enumerate}
This should complete an informal introduction to higher category theory with many references interesting in this direction.
\\
The last two sections are closely related to homotopy theory. And since homotopy theory is very important in our context we pay more attention to the subject than the ordinary introductory texts on category theory. Let us briefly discuss the topics we mean.
\begin{enumerate}
\item[$\bullet$]
Simplicial sets in section \ref{sec:sset}. In this section we discuss how simplicial sets play a major role in homotopy theory (even more than one would expect as from a better version of simplicial complexes). In particular, we discuss how to use them in the classification of principal bundles by homotopy classes and cohomology ({\glqq}calculated{\grqq} with \v{C}ech cohomology).
\item[$\bullet$]
Fibrations in section \ref{sec:fibration}. For an algebraic topologist or even classical homotopy theorist it is obvious that this has to do with homotopy theory. We discuss a very general idea of fibrations and their duals (cofibrations) encompassing the notions from classical homotopy theory. The highlight\footnote{not only of this section} is astonishingly a pretty formal definition of {\glqq}stack{\grqq}. These homotopy-theortic versions of sheaves\footnote{defined somewhere in section \ref{sec:uni}} seem to be the proposed mathmatical objects containing the information of physical fields in quantum field theory obeying the principle of locality.
\end{enumerate}
We hope we made it to give a reasonable overview of this chapter to save you from missing the forest for the trees.
