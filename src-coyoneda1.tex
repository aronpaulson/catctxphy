This subsubsection is about a consequence of the so-called Co-Yoneda lemma that owes its name to being some kind of dual to the actual yoneda lemma \ref{lem:yoneda}. The Co-Yoneda lemma is discussed and formulated in subsubsection \ref{sec:coyoneda2} after we have developed some more machinery. The consequence we discuss in this subsubsection with more elementary means is the so-called density theorem.  What it actually says is that any presheaf $P$ on $\mathbf{C}$ is the colimit of contravariant hom-functors indexed by the category of colements of $P$. That is, the representable presheaves on $\mathbf{C}$ as generalized objects are enough to build any generalized object in the presheaf category by gluing the representable generalized objects together. Gluing objects together is the very idea of a colimit as should have become clear from this subsection \ref{sec:limit}. This will become even clearer when the objects have a geometric meaning as in section \ref{sec:sset} where they are considered as triangles. One can also interpret colimit a bit as a kind of limiting process similar to limits in analysis. By that we mean that presheaves are {\glqq}approximated{\grqq} well enough by representable presheaves which is why people consider density theorem a suitable name for the following theorem.
\\
\begin{thm}[Density]
\label{thm:density}
Let $P \colon \mathbf{C}^{\mathrm{op}} \rightarrow \mathbf{Set}_{\mathcal{U}}$ be a presheaf on a $\mathcal{U}$-small $\mathbf{C}$ and define a projection functor on the category of coelements of $P$ by
\begin{align*}
  \pi_{P}
  \colon
  \int_{\mathbf{C}}^{\prime}
  P
  &\rightarrow
  \mathbf{C}
  \\
  (X,x)
  &\mapsto
  X
  \\
  f_{21}^{\mathrm{op}}
  &\mapsto
  f_{12}
\end{align*}
Further compose this functor with the Yoneda functor as
\begin{align*}
  \mathrm{y}_{P}
  &:=
  \mathrm{y}_{\mathbf{C}}
  \circ
  \pi_{P}
\end{align*}
and define a function
\begin{align*}
  \mathsf{C}_{P}^{\prime}
  \colon
  \mathrm{ob}_{\int_{\mathbf{C}}^{\prime}P}
  &\rightarrow
  \bigcup_{X \in \mathrm{ob}_{\mathbf{C}^{\mathrm{op}}}}
  \mathrm{mor}_{\mathbf{Set}^{\mathbf{C}^{\mathrm{op}}}}
  \left(
    \mathrm{y}_{\mathbf{C}}(X),
    P
  \right)
  \\
  (X,x)
  &\mapsto
  \mathsf{Y}(P,X)^{-1}(x)
\end{align*}
where $\mathsf{Y}(P,X)$ denotes the Yoneda isomorphism. Then $\mathsf{C}_{P}^{\prime}$ is a colimiting cocone for $\mathrm{y}_{P}$ with apex $P$. Hence
\begin{align*}
  P
  &\cong
  \varinjlim_{\int_{\mathbf{C}}^{\prime}P}(\mathrm{y}_{P})
\end{align*}
\end{thm}
\begin{prf}
As always in such situations we proceed in two steps. We first show that $\mathsf{C}_{P}^{\prime}$ is a cocone and then that it is colimiting.
\begin{description}
\item[Step 1]
We will incorporate the Yoneda lemma \ref{lem:yoneda} to show that $\mathsf{C}_{P}^{\prime}$ is a cocone. Consider
\begin{align*}
  (X_{1},x_{1}),
  (X_{2},x_{2})
  \in
  \mathrm{ob}_{\int_{\mathbf{C}}^{\prime}P}
\end{align*}
Then
\begin{align*}
  x_{1}
  &=
  P(f_{21}^{\mathrm{op}})(x_{2})
\end{align*}
holds for all
\begin{align*}
  f_{21}^{\mathrm{op}}
  \in
  \mathrm{mor}_{\int_{\mathbf{C}}^{\prime}P}
  \left(
    (X_{1},x_{1}),
    (X_{2},x_{2})
  \right)
\end{align*}
Applying the Yoneda isomorphism and using its naturality then yields the equivalent equation
\begin{align*}
  \mathsf{Y}(P,X_{1})^{-1}(x_{1})
  &=
  \left(
    \mathsf{Y}(P,X_{1})^{-1}
    \circ
    P(f_{21}^{\mathrm{op}})
    \circ
    \mathsf{Y}(P,X_{2})
    \circ
    \mathsf{Y}(P,X_{2})^{-1}
  \right)
  (x_{2})
  \\
  &=
  \mathrm{prob}_{\mathrm{y}}(\mathrm{id}_{P},f_{21}^{\mathrm{op}})
  \left(
    \mathsf{Y}(P,X_{2})^{-1}(x_{2})
  \right)
  \tag{NT}
  \\
  &=
  \mathsf{Y}(P,X_{2})^{-1}(x_{2})
  \circ
  \mathrm{y}_{\mathbf{C}}(f_{21}^{\mathrm{op}})
\end{align*}
Hence by definition of $\mathsf{C}_{P}^{\prime}$ this implies a commutative diagram
\[
\begin{tikzcd}[sep=huge]
  \mathrm{y}_{\mathbf{C}}(X_{1})
  \arrow{rr}{\mathrm{y}_{\mathbf{C}}(f_{21}^{\mathrm{op}})}
  \arrow[swap]{dr}{\mathsf{C}_{P}^{\prime}(X_{1},x_{1})}
  &
  &
  \mathrm{y}_{\mathbf{C}}(X_{2})
  \arrow{dl}{\mathsf{C}_{P}^{\prime}(X_{2},x_{2})}
  \\
  &
  P
  &
\end{tikzcd}
\]
finally showing that $\mathsf{C}_{P}^{\prime}$ is a cocone. 
\\
\item[Step 2]
Let $\mathsf{C}^{\prime}$ denote any cocone to $\mathrm{y}_{P}$ with apex $P^{\backprime}$. We can then apply the Yoneda isomorphism $\mathsf{Y}(P^{\backprime},X)$ to $\mathsf{C}^{\prime}(X,x)$ to get the corresponding element in $P^{\backprime}(X)$ for each
\begin{align*}
  (X,x)
  \in
  \mathrm{ob}_{\int_{\mathbf{C}}^{\prime}P}
\end{align*}
Let us denote this by
\begin{align*}
  \mathrm{y}_{(X,x)}^{\backprime}
  &:=
  \mathsf{Y}(P^{\backprime},X)
  \left(
    \mathsf{C}^{\prime}(X,x)
  \right)
\end{align*}
We have to find a unique natural transformation $\mathsf{T}$ from $P$ to $P^{\backprime}$ such that
\begin{align*}
  \mathsf{C}^{\prime}(X,x)
  &=
  \mathsf{T}
  \circ
  \mathsf{C}_{P}^{\prime}(X,x)
\end{align*}
for all
\begin{align*}
  (X,x)
  \in
  \mathrm{ob}_{\int_{\mathbf{C}}^{\prime}P}
\end{align*}
We propose defining $\mathsf{T}$ by mapping an object $X$ to
\begin{align*}
  \mathsf{T}(X)
  \colon
  P(X)
  &\rightarrow
  P^{\backprime}(X)
  \\
  x
  &\mapsto
  \mathrm{y}_{(X,x)}^{\backprime}
\end{align*}
Since $\mathsf{Y}$ is natural, for arbitrary $f_{21}^{\mathrm{op}}$ and $x_{1} \in P(X_{1})$
the equations
\begin{align*}
  \left(
    P^{\backprime}(f_{21}^{\mathrm{op}})
    \circ
    \mathsf{T}(X_{2})
  \right)
  (x_{2})
  &=
  P^{\backprime}(f_{21}^{\mathrm{op}})
  \left(
    \mathrm{y}_{(X_{2},x_{2})}^{\backprime}
  \right)
  \\
  &=
  P^{\backprime}(f_{21}^{\mathrm{op}})
  \left(
    \mathsf{Y}(P^{\backprime},X_{2})
    \left(
      \mathsf{C}^{\prime}(X_{2},x_{2})
    \right)
  \right)
  \\
  &=
  \mathsf{Y}(P^{\backprime},X_{1})
  \left(
    \mathrm{prob}_{\mathrm{y}}
    \left(
      \mathrm{id}_{P^{\backprime}},
      f_{21}^{\mathrm{op}}
    \right)
    \left(
      \mathsf{C}^{\prime}(X_{2},x_{2})
    \right)
  \right)
  \tag{NT}
  \\
  &=
  \mathsf{Y}(P^{\backprime},X_{1})
  \left(
    \mathsf{C}^{\prime}(X_{2},x_{2})
    \circ
    \mathrm{y}_{\mathbf{C}}(f_{21}^{\mathrm{op}})
  \right)
  \\
  &=
  \mathsf{Y}(P^{\backprime},X_{1})
  \left(
    \mathsf{C}^{\prime}
    \left(
      X_{1},
      \left(
        P(f_{21}^{\mathrm{op}})
      \right)
      (x_{2})
    \right)
  \right)
  \\
  &=
  \mathrm{y}_{(X_{1},(P(f_{21}^{\mathrm{op}}))(x_{2}))}^{\backprime}
  \\
  &=
  \left(
    \mathsf{T}(X_{1})
    \circ
    P(f_{21}^{\mathrm{op}})
  \right)
  (x_{2})
\end{align*}
prove naturality of $\mathsf{T}$. Moreover the equation
\begin{align*}
  \mathsf{T}
  \circ
  \mathsf{C}_{P}^{\prime}(X,x)
  &=
  \mathsf{C}^{\prime}(X,x)
\end{align*}
holds for all
\begin{align*}
  (X,x)
  \in
  \mathrm{ob}_{\int_{\mathbf{C}}^{\prime}P}
\end{align*}
since
\begin{align*}
  \mathsf{Y}(P^{\backprime},X)
  \left(
    \mathsf{T}
    \circ
    \mathsf{C}_{P}^{\prime}(X,x)
  \right)
  &=
  \mathsf{Y}(P^{\backprime},X)
  \left(
    \mathsf{T}
    \circ
    \mathsf{Y}(P,X)^{-1}(x)
  \right)
  \tag{NT}
  \\
  &=
  (\mathsf{T}(X))(x)
  \\
  &=
  \mathrm{y}_{(X,x)}^{\backprime}
  \\
  &=
  \mathsf{Y}(P^{\backprime},X)
  \left(
    \mathsf{C}^{\prime}(X,x)
  \right)
\end{align*}
is valid again due to the naturality of $\mathsf{Y}$.
\\
What is left to prove is that $\mathsf{T}$ constructed so is unique. For this purpose let $\mathsf{O} \colon P \Rightarrow P^{\backprime}$ such that
\begin{align*}
  \mathsf{C}^{\prime}(X,x)
  &=
  \mathsf{O}
  \circ
  \mathsf{C}_{P}^{\prime}(X,x)
\end{align*}
for all
\begin{align*}
  (X,x)
  \in
  \mathrm{ob}_{\int_{\mathbf{C}}^{\prime}P}
\end{align*}
We then get for all such $(X,x)$
\begin{align*}
  (\mathsf{T}(X))(x)
  &=
  \mathsf{Y}(P^{\backprime},X)
  \left(
    \mathsf{C}^{\prime}(X,x)
  \right)
  \\
  &=
  \mathsf{Y}(P^{\backprime},X)
  \left(
    \mathsf{O}
    \circ
    \mathsf{C}_{P}^{\prime}(X,x)
  \right)
  \\
  &=
  \mathsf{Y}(P^{\backprime},X)
  \left(
    \mathsf{O}
    \left(
      \mathsf{Y}(P,X)^{-1}(x)
    \right)
  \right)
  \tag{NT}
  \\
  &=
  (\mathsf{O}(X))(x)
\end{align*}
showing that $\mathsf{T} = \mathsf{O}$.
\end{description}
Now we are done.
\\
\phantom{proven}
\hfill
$\square$
\end{prf}
This was a direct prove of the density theorem. But in subsubsection \ref{sec:coyoneda2} we will infer it in a more general context.
