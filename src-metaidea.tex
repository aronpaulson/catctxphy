There are many ideas around (higher) category theory which are not always really formal. These ideas are often only expressed in a more or less precise english - or any other language with enough expressive power. This is to say that they only exist on the meta level at this point of the discussion. Of course, one would like to formalize these ideas to make them mathematically rigorous. But this has not been achieved yet for all the important ideas presented in this section albeit for some. We splitted these section into two subsections:
\begin{enumerate}
\item[$\bullet$]
Internalization
\item[$\bullet$]
Higher Level Structure
\end{enumerate}
While the first subsection takes place in a $1$-categorical context the second one is all about the higher categorical one. As you will see the idea of internalization makes also sense in a higher categorical context. The reason for an own section for it is rather its importance to us here. We will concentrate on topoi and principal bundles there besides the actual idea. The section about higher level structures is much about some more intuition behind higher category theory and contains the important ideas of
\begin{enumerate}
\item[$\bullet$]
Categorification
\item[$\bullet$]
Oidification
\end{enumerate}
which we will then apply to a monoid in a subsubsection. In this subsubsection we will briefly discuss the further idea of
\begin{enumerate}
\item[$\bullet$]
Enrichment
\end{enumerate}
A last thing we want to explicitly remark here is that it also contains the idea of operads which is a nice tool to track the higher structure of higher categories and has its origin in homotopy theory.
