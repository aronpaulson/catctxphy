There is a generic but somewhat technical category which often appears throughout mathematics. This category is called comma category for historical reason and recognizing this category when it emerges usually allows a neater presentation. After defining this category precisely we consider a few special cases at the end of this subsection. These special cases will give a taste of what neater presentation means.
\\
Given functors $F_{\alpha\omega},F_{\beta\omega}$ a category $\mathbf{C}$ is called the \textbf{comma category (of $F_{\alpha\omega}$ and $F_{\beta\omega}$)} if elements of $\mathrm{ob}_{\mathbf{C}}$ are exactly the $3$-tuples $(X^{\alpha},X^{\beta},w)$ where
\begin{align*}
  w
  \in
  \mathrm{mor}_{\mathbf{C}_{\omega}}
  \left(
    F_{\alpha\omega}(X^{\alpha}),
    F_{\beta\omega}(X^{\beta})
  \right)
\end{align*}
and if the elements of
\begin{align*}
  \mathrm{mor}_{\mathbf{C}}
  \left(
    (X_{1}^{\alpha},X_{1}^{\beta},w_{1}),
    (X_{2}^{\alpha},X_{2}^{\beta},w_{2})
  \right)
\end{align*}
are exactly the tuples $(f_{12}^{\alpha},f_{12}^{\beta})$ such that the diagram 
\[
\begin{tikzcd}[sep=large]
  F_{\alpha\omega}(X_{1}^{\alpha})
  \arrow{r}{F_{\alpha\omega}(f_{12}^{\alpha})}
  \arrow[swap]{d}{w_{1}}
  &
  F_{\alpha\omega}(X_{2}^{\alpha})
  \arrow{d}{w_{2}}
  \\
  F_{\beta\omega}(X_{1}^{\beta})
  \arrow{r}{F_{\beta\omega}(f_{12}^{\beta})}
  &
  F_{\beta\omega}(X_{2}^{\beta})
\end{tikzcd}
\]
commutes while composition is defined by
\begin{align*}
  (f_{23}^{\alpha},f_{23}^{\beta})
  \circ_{\mathbf{C}}
  (f_{12}^{\alpha},f_{12}^{\beta})
  &:=
  (f_{23}^{\alpha} \circ f_{12}^{\alpha},f_{23}^{\beta} \circ f_{12}^{\beta})
\end{align*}
The comma category of $F_{\alpha\omega}$ and $F_{\beta\omega}$ is usually denoted $(F_{\alpha\omega} \downarrow F_{\beta\omega})$. That $(F_{\alpha\omega} \downarrow F_{\beta\omega})$ is really a category becomes apparent if one identifies $(\mathrm{id}_{X^{\alpha}},\mathrm{id}_{X^{\beta}})$ as the identity. From the construction one sees that $(F_{\alpha\omega} \downarrow F_{\beta\omega})$ is not just a mere category which is why the dual formula of the formula represented by $(F_{\alpha\omega} \downarrow F_{\beta\omega})$ is not $(F_{\alpha\omega} \downarrow F_{\beta\omega})^{\mathrm{op}}$ but rather
\begin{align*}
  \left(
    F_{\alpha\omega}^{\mathrm{op}}
    \downarrow
    F_{\beta\omega}^{\mathrm{op}}
  \right)^{\mathrm{op}}
\end{align*}
Consequently,
\begin{align*}
  \left(
    F_{\alpha\omega}^{\mathrm{op}}
    \downarrow
    F_{\beta\omega}^{\mathrm{op}}
  \right)^{\mathrm{op}}
\end{align*}
is called \textbf{cocomma category (of $F_{\alpha\omega}$ and $F_{\beta\omega}$)}. It is clear that
\begin{align*}
  \left(
    F_{\alpha\omega}^{\mathrm{op}}
    \downarrow
    F_{\beta\omega}^{\mathrm{op}}
  \right)^{\mathrm{op}}
  &\cong
  (F_{\beta\omega} \downarrow F_{\alpha\omega})
\end{align*}
as a category. This isomorphism is often used tacitly in practice in the sense that $(F_{\beta\omega} \downarrow F_{\alpha\omega})$ represents a formula which holds if and only if the formula represented by
\begin{align*}
  \left(
    F_{\alpha\omega}^{\mathrm{op}}
    \downarrow
    F_{\beta\omega}^{\mathrm{op}}
  \right)^{\mathrm{op}}
\end{align*}
holds. Thus we can equivalently define:
\begin{align*}
  (F_{\beta\omega} \downarrow F_{\alpha\omega})
\end{align*}
is called the \textbf{cocomma category (of $F_{\alpha\omega}$ and $F_{\beta\omega}$)}. And for the sake of notational simplicity we shall use the latter definition of cocomma category hereafter.
\\
We now turn to the promised special cases from practice which are all isomorphic to comma categories:
\begin{enumerate}
\item[(1)]
The comma category $(\mathrm{id}_{\mathbf{C}} \downarrow \mathrm{id}_{\mathbf{C}})$ is called the \textbf{category of bundles (of $\mathbf{C}$)}. The objects here are $3$-tuples $(X_{1},X_{2},p_{12})$ with
\begin{align*}
  p_{12}
  &\in
  \mathrm{mor}_{\mathbf{C}}(X_{1},X_{2})
\end{align*}
while a morphism from $(X_{1},X_{2},p_{12})$ to $(X_{3},X_{4},p_{34})$ is a pair of morphisms
\begin{align*}
  (f_{13},f_{24})
  &\in
  \mathrm{mor}_{\mathbf{C}}(X_{1},X_{3})
  \times
  \mathrm{mor}_{\mathbf{C}}(X_{2},X_{4})
\end{align*}
such that the diagram
\[
\begin{tikzcd}[sep=large]
  X_{1}
  \arrow{r}{f_{13}}
  \arrow[swap]{d}{p_{12}}
  &
  X_{3}
  \arrow{d}{p_{34}}
  \\
  X_{2}
  \arrow{r}{f_{24}}
  &
  X_{4}
\end{tikzcd}
\]
commutes. The terminology is due to the classical case when $\mathbf{C} = \mathbf{Top}$ since then an object of
\begin{align*}
  (\mathrm{id}_{\mathbf{Top}} \downarrow \mathrm{id}_{\mathbf{Top}})
\end{align*}
is simply a bundle as traditionally defined in topology while a morphism is bundle map. Hence we call an object $(X_{1},X_{2},p_{12})$ of the category of bundles of $\mathbf{C}$ a \textbf{bundle (of $\mathbf{C}$)} while we call a morphism $(f_{13},f_{24})$ from $(X_{1},X_{2},p_{12})$ to $(X_{3},X_{4},p_{34})$ a \textbf{bundle map (from $(X_{1},X_{2},p_{12})$ to $(X_{3},X_{4},p_{34})$ of $\mathbf{C}$)}. Moreover for a bundle $(X_{1},X_{2},p_{12})$ we call $X_{1}$ the \textbf{total space (of $p_{12}$)} and $X_{2}$ the \textbf{base space (of $p_{12}$)}. Next we define an obviously isomorphic category but nevertheless keep the bundle bundle terminology: $\mathbf{C}_{\rightarrow}$ is called \textbf{arrow category (of $\mathbf{C}$)} if
\begin{align*}
  \mathrm{ob}_{\mathbf{C}_{\rightarrow}}
  &=
  \mathrm{Mor}_{\mathbf{C}}
\end{align*}
and
\begin{align*}
  \mathrm{mor}_{\mathbf{C}_{\rightarrow}}(p_{12},p_{34})
  &=
  \mathrm{mor}_{(\mathrm{id}_{\mathbf{C}} \downarrow \mathrm{id}_{\mathbf{C}})}
  \left(
    (X_{1},X_{2},p_{12}),
    (X_{3},X_{4},p_{34})
  \right)
\end{align*}
for all $X_{1},X_{2},X_{3},X_{4}$ and all
\begin{align*}
  p_{12}
  &\in
  \mathrm{mor}_{\mathbf{C}}(X_{1},X_{2})
  \\
  p_{34}
  &\in
  \mathrm{mor}_{\mathbf{C}}(X_{3},X_{4})
\end{align*}
It is clear that $(\mathrm{id}_{\mathbf{C}} \downarrow \mathrm{id}_{\mathbf{C}})$ is isomorphic to $\mathbf{C}_{\rightarrow}$ and so we can consider it structurally as the same. That is, we will often prefer $\mathbf{C}_{\rightarrow}$ over $(\mathrm{id}_{\mathbf{C}} \downarrow \mathrm{id}_{\mathbf{C}})$.
\item[(2)]
The comma category $(\mathrm{id}_{\mathbf{C}} \downarrow \mathrm{c}_{X})$ is called the \textbf{category of bundles (over $X$ of $\mathbf{C}$)}. The objects here are $3$-tuples $(X_{1},X,p_{1})$ with
\begin{align*}
  p_{1}
  &\in
  \mathrm{mor}_{\mathbf{C}}(X_{1},X)
\end{align*}
while a morphism from $(X_{1},X,p_{1})$ to $(X_{2},X,p_{2})$ is a pair of morphisms
\begin{align*}
  (f_{12},\mathrm{id}_{X})
  &\in
  \mathrm{mor}_{\mathbf{C}}(X_{1},X_{2})
  \times
  \mathrm{mor}_{\mathbf{C}}(X,X)
\end{align*}
such that the diagram
\[
\begin{tikzcd}[sep=large]
  X_{1}
  \arrow{rr}{f_{12}}
  \arrow[swap]{dr}{p_{1}}
  &
  &
  X_{2}
  \arrow{dl}{p_{2}}
  \\
  &
  X
  &
\end{tikzcd}
\]
commutes, the object $X$ being fixed all the time. It is clear that $(\mathrm{id}_{\mathbf{C}} \downarrow \mathrm{c}_{X})$ is a subcategory of $(\mathrm{id}_{\mathbf{C}} \downarrow \mathrm{id}_{\mathbf{C}})$. The terminology is again due to the classical case when $\mathbf{C} = \mathbf{Top}$ since then an object of
\begin{align*}
  (\mathrm{id}_{\mathbf{Top}} \downarrow \mathrm{c}_{X})
\end{align*}
is simply a bundle over $X$ as traditionally defined in topology while a morphism is a bundle map of bundles over the same base space. Hence we call an object $(X_{1},X,p_{1})$ of the category of bundles over $X$ of $\mathbf{C}$ a \textbf{bundle (over $X$ of $\mathbf{C}$)} while we call a morphism $(f_{12},\mathrm{id}_{X})$ from $(X_{1},X,p_{1})$ to $(X_{2},X,p_{2})$ a \textbf{bundle map (from $(X_{1},X,p_{1})$ to $(X_{2},X,p_{2})$ over $X$ of $\mathbf{C}$)}. We in turn define an obviously isomorphic category but nevertheless keep the bundle bundle terminology: $\mathbf{C} \slash X$ is called\footnote{the slice notated as $\slash$ can in fact be understood as a division as can be seen after some more category theory and understanding \cite{c55c71e8} (the point is that $\mathrm{id}_{X}$ is a so called terminal object)} \textbf{slice category (over $X$ of $\mathbf{C}$)} if
\begin{align*}
  \mathrm{ob}_{\mathbf{C} \slash X}
  =
  \bigcup_{X_{1} \in \mathrm{ob}_{\mathbf{C}}}
  \mathrm{mor}_{\mathbf{C}}(X_{1},X)
\end{align*}
and
\begin{align*}
  \mathrm{mor}_{\mathbf{C} \slash X}(p_{1},p_{2})
  &=
  \left\lbrace
      f_{12}
      \in
      \mathrm{mor}_{\mathbf{C}}(X_{1},X_{2})
    \,
    \vert
    \,
      p_{1}
      =
      p_{2}
      \circ
      f_{12}
  \right\rbrace
\end{align*}
It is clear that $(\mathrm{id}_{\mathbf{C}} \downarrow \mathrm{c}_{X})$ is isomorphic to $\mathbf{C} \slash X$ and so we can consider it structurally as the same. That is, we will mostly prefer $\mathbf{C} \slash X$ over $(\mathrm{id}_{\mathbf{C}} \downarrow \mathrm{c}_{X})$.
\item[(3)]
Let $F \colon \mathbf{C} \rightarrow \mathbf{Set}$ be a functor. The comma category $(\mathrm{c}_{\lbrace \emptyset \rbrace} \downarrow F)$ has objects $(\lbrace \emptyset \rbrace,X,w)$ with
\begin{align*}
  w
  &\in
  \mathrm{mor}_{\mathbf{C}}(\lbrace \emptyset \rbrace,F(X))
\end{align*}
while a morphism from $(\lbrace \emptyset \rbrace,X_{1},w_{1})$ to $(\lbrace \emptyset \rbrace,X_{2},w_{2})$ is a pair of morphisms
\begin{align*}
  (\mathrm{id}_{\lbrace \emptyset \rbrace},f_{12})
  &\in
  \mathrm{mor}_{\mathbf{C}}(\lbrace \emptyset \rbrace,\lbrace \emptyset \rbrace)
  \times
  \mathrm{mor}_{\mathbf{C}}(X_{1},X_{2})
\end{align*}
such that the diagram
\[
\begin{tikzcd}[sep=large]
  &
  \lbrace
    \emptyset
  \rbrace
  \arrow[swap]{dl}{w_{1}}
  \arrow{dr}{w_{2}}
  &
  \\
  F(X_{1})
  \arrow{rr}{F(f_{12})}
  &
  &
  F(X_{2})
\end{tikzcd}
\]
commutes. It is clear that if we take any one element set instead of $\lbrace \emptyset \rbrace$ we get an isomorphic category. Only the structure of a one element set matters here since its only purpose is to choose an element from $F(X)$ via $w$ - namely $w(\emptyset)$. This idea is a very important one in category theory an particularly in topos theory. We will come back to it as the text goes on. Anyways, it is clear that $(\lbrace \emptyset \rbrace,X,w)$ and $(X,w(\emptyset))$ correspond to each other while a morphism from $(\lbrace \emptyset \rbrace,X_{1},w_{1})$ to $(\lbrace \emptyset \rbrace,X_{2},w_{2})$ in the sense above is nothing but a morphism $f_{12} \in \mathrm{mor}_{\mathbf{C}}(X_{1},X_{2})$ such that
\begin{align*}
  F(f_{12})(w_{1}(\emptyset))
  &=
  w_{2}(\emptyset)
\end{align*}
Hence the category $\int_{\mathbf{C}}F$ called \textbf{category of elements (of $F$)} with objects the tuples $(X,x)$ where $X \in \mathrm{ob}_{\mathbf{C}}$ and $x \in F(X)$ and morphisms
\begin{align*}
  \mathrm{mor}_{\int_{\mathbf{C}}F}
  \left(
    (X_{1},x_{1}),
    (X_{2},x_{2})
  \right)
  &:=
  \left\lbrace
      f_{12}
      \in
      \mathrm{mor}_{\mathbf{C}}(X_{1},X_{2})
    \,
    \vert
    \,
      F(f_{12})(x_{1})
      =
      x_{2}
  \right\rbrace
\end{align*}
is isomorphic to $(\mathrm{c}_{\lbrace \emptyset \rbrace} \downarrow F)$. That is, we will actually always prefer $\int_{\mathbf{C}}F$ over $(\mathrm{c}_{\lbrace \emptyset \rbrace} \downarrow F)$. $\int_{\mathbf{C}}F$ seems notationally just way simpler.
\\
There is a dual notion of the above better suited to the needs of a presheaf $F \colon \mathbf{C}^{\mathrm{op}} \rightarrow \mathbf{Set}$ by using the cocomma category of $\mathrm{c}_{\lbrace \emptyset \rbrace}$ and $F$
\begin{align*}
  (F \downarrow \mathrm{c}_{\lbrace \emptyset \rbrace})
\end{align*}
By the dual reasoning this category is isomorphic to the the category $\int_{\mathbf{C}}^{\prime}F$ called \textbf{category of coelements (of $F$)} with objects the tuples $(X,x)$ where $X \in \mathrm{ob}_{\mathbf{C}}$ and $x \in F(X)$ and morphisms
\begin{align*}
  \mathrm{mor}_{\int_{\mathbf{C}}^{\prime}F}
  \left(
    (X_{1},x_{1}),
    (X_{2},x_{2})
  \right)
  &:=
  \left\lbrace
      f_{21}^{\mathrm{op}}
      \in
      \mathrm{mor}_{\mathbf{C}^{\mathrm{op}}}(X_{2},X_{1})
    \,
    \vert
    \,
      F(f_{21}^{\mathrm{op}})(x_{2})
      =
      x_{1}
  \right\rbrace
\end{align*}
Of course, we actually always prefer $\int_{\mathbf{C}}^{\prime}F$ over $(F \downarrow \mathrm{c}_{\lbrace \emptyset \rbrace})$.
\end{enumerate}
