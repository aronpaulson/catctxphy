%\nocite{c55c71e8}
%\nocite{e837ef86}
%\nocite{a565d200}
%\nocite{dc6f686f}
%\nocite{1ba1603e}
%\nocite{52fbba46}
%\nocite{8b5861fc}
%\nocite{wiki-nlab0000}
%\nocite{wiki-pedia0en}
The title of these notes is quite explicit on what these notes are about: category theory in the context of the geometry of physics. The title is borrowed from Riehl's book {\glqq}Category Theory in Context{\grqq} (\cite{52fbba46}).\footnote{in fact, we can really recommend it} The idea of the notes is being an introductory text on category theory and how to apply it to modern physics. By modern physics we mean things like {\glqq}gauge theory{\grqq}. Eventually, these notes can be regarded as the starting point of a {\glqq}directed path{\grqq} with terminus \cite{a565d200}.
\\\\
Actually, the bar is low to read these notes. While the ideal reader would be a mathematical physicist (let's just say on an advanced undergraduate level, at least) we claim that a good part of these notes is understandable with way less knowledge on mathematics. This is because category theory can be regarded as being at the very beginning of mathematics. So the target group is anybody with some basic mathematical knowledge (we will say {\glqq}what{\grqq} in a moment) possibly with an interest in modern physics. So what do we require you to know? Well, there are several levels on which one can read these notes.
\begin{enumerate}
\item[(1)]
If you want to understand the ordinary category theory part only, then there is not much you are required to know already. It suffices to be familiar with very basic mathematics, which we claim someone doing mathematics on university for one year is:
\begin{enumerate}
\item[$\bullet$]
Reasoning in classical logic about mathematical objects. This in particular includes an understanding of first-order theories\footnote{you do not have to know what this precisely is} on an intuitive level. That is, you should be able to read and understand formulas such as $\exists x \forall y(y \in x \Rightarrow \ldots)$ and you should have a feeling for what a mathematical proof is.
\item[$\bullet$]
Set theory and some fundamental constructions in set theory. You are not required to know formal set theory (such as ZFC) but only what one can do with sets in ordinary mathematics on an intuitive level. In particular, you should know about unions, intersections, ordered pairs and perhaps some other fundamental stuff we forgot to list here.
\item[$\bullet$]
Relations and particularly equivalence relations.
\item[$\bullet$]
Functions. Here we mean functions between sets. Note that we might terminologically differ functions from {\glqq}assignments{\grqq}.\footnote{this is not standard (sometimes people speak of definable functions in this context)} In our terminology assignments not only map the elements of a set - an object of the mathemtical universe - to elements of some further set, but (a part of) the objects in the mathematical universe to objects of the mathematical universe.
\item[$\bullet$]
Ordered sets. We require knowledge about preordered sets, partially ordered sets, totally ordered sets and well-ordered sets.
\end{enumerate}
This is theoretically enough for most of the notes. Yet, it is certainly better to know more about mathematics for a deeper understanding. Since category theory can be regarded as being at the beginning of mathematics and since category theory is a first-order theory it is helpful to have a certain understanding of what mathematics actually is (there are different ideas on that). But be warned: in our experience a book about mathematical logic raises in some sense more questions than it answers and you have to work a bit to get kind of a full picture (if there is one at all). Yet it improves the understanding of what one actually does in mathematics. We are not aware of a single source explaining what we mean and honestly we should write a prequel to these notes. But for now you are on your own to learn about it. However, you can take action on another level to improve the understanding of these notes. Many of our examples originate in certain traditional fields of mathematics: algebra and topology. A proper understanding of these fields will help you understand the examples and thus category theory better. We will discuss most things from algebra and topology we need for reasons of self-containment. But do not expect a too pedagogical discussion. So it would be good to already know about algebra, topology and perhaps a bit on the foundations of mathematics.
\item[(2)]
If you want to understand the whole category theory part there is no getting round homotopy theory. Homotopy theory is deeply linked to category theory and higher category theory. This is one of the major aspects of these notes as you will very soon realize. Also, cohomology will play a big role in these notes. We do not want to spoil too much here (and bloat this introduction) so we only say a perfectly prepared reader is already familiar with some traditional algebraic topology.
\item[(3)]
If you also want to understand the (physical) context then you should know a bit about basic physics and differential geometry on the mathematical side (in addition to the basic mathematics from above). It is presumably not necessary to be an expert, neither in physics nor differential geometry. We tried to keep it intuitive as far as possible. So you can give it a try even if you do not know much about these things. Still, we think one should know a bit about these things. What you can try is first reading the category part and only then the context.
\end{enumerate}
Note that we propose references along the notes on some of the stuff we discussed as requirements on the reader. Moreover note that we sometimes write terms in \textit{italics} when the reader is not expected to already know these. Last but not least we also want to explicitly emphasize that you should just skip things you do not understand. You should then still be able to understand what is happening. At least, this is our intention which we hopefully implemented well enough.
\\\\
After discussing the requirements on the reader let us discuss the content of the notes a bit. We have four chapters and one appendix. The first chapter is the introduction and it apparently doesn't make sense to describe its content here. For the other chapters and the appendix we only give a rough description in this introduction since in the introduction of the single chapters we will say what the chapter and particularly the sections therein are about. A more thorough discussion of the content of each section is then in the introduction of the respective section.
\\
We have one main chapter - chapter \ref{chap:cattg} - which constitutes the major part of these notes. It's title is {\glqq}Category Theory Interpreted in TG{\grqq} and this is essentially what we do there. We develop category theory in let's just say ordinary mathematics. With ordinary mathematics we actually mean the way you are used to do mathematics. The only difference (which may be new to you) is that we use {\glqq}Grothendieck universes{\grqq}. This is because in category theory one often has to do with large collections of things and Grothendieck universes are a neat way to handle these (way better than the classes in von-Neumann Bernays G\"{o}del (abbr. NBG) or even informal classes). Appendix \ref{chap:tg} is an overview on what we mean by TG and particularly Grothendieck universes. But let us stress that that chapter \ref{chap:cattg} is not only about category theory but to a not so small extent about higher category theory and homotopy theory. Chapter \ref{chap:cattg} is then embraced by the two other chapters: the second chapter \ref{chap:initcontext} and the last chapter \ref{chap:termcontext}. These are called {\glqq}Initial Context{\grqq} and {\glqq}Terminal Context{\grqq}, respetively.\footnote{after reading these notes you can think about if these titles are more than a pun} While this seems as if the initial context chapter \ref{chap:initcontext} is about the physical motivation and problems to resolve with category theory and the terminal context chapter \ref{chap:termcontext} as an evaluation how well we accomplished this mission this is not the whole truth. The initial context chapter \ref{chap:initcontext} is, in fact, about the geometry of physics in traditional mathematics (more or less differential geometry) and problems to be resolved with category theory. However, the terminal context chapter \ref{chap:termcontext} is not only an evaluation of the notes. It is rather about that we only presented the peak of the iceberg. We illustrate this by a very general idea of cohomology and how to calculate it in the light of gauge theory. Last we propose a way to see the whole {\glqq}iceberg{\grqq} in \cite{a565d200}.
\\
There are a few sources the content of these notes is based on which we do not want to keep back. The obvious one is wikipedia \cite{wiki-pedia0en}. Probably anybody uses this as general purpose source and so do we. But for (higher) category theory
 and many things around this there is another such thing: the nLab \cite{wiki-nlab0000}. If you are seriously interested in what we do here then this is your friend. The category theory we do is based on the following three sources:
\begin{enumerate}
\item[$\bullet$]
Mac Lane's traditional book \cite{e837ef86} (not so pedagogical but covers many topics and is possibly the most famous one of the three)
\item[$\bullet$]
Leinster's introductory book \cite{dc6f686f} (we didn't use it much but it kind of left an impression)
\item[$\bullet$]
Riehl's modern book \cite{52fbba46} (which we used most of the three)
\end{enumerate}
If you recognize some wisdom on category theory in these notes it is arguably Mac Lane's and Moerdijk's book on sheaves and topoi \cite{c55c71e8} you have to blame. Last but not least we should not forget about homotopy theory:
\begin{enumerate}
\item[$\bullet$]
Hatcher's well known algebraic topology book \cite{8b5861fc} for a traditional view
\item[$\bullet$]
The Univalent Foundations Program book on homotopy type theory \cite{1ba1603e} for a modern synthetic approach
\end{enumerate}
Most of these sources are heavily cited in these notes.
\\\\
Now we have to talk about the style of this text a bit. Our focus is on intuition but also on being precise as far as possible. We tried to motivate anything on an intuitive level on the one hand while additionally be as formal as needed on the other. This is because being fully formal is unambiguous but hard to understand for humans. So we think it is better to babble much around things in a natural language in addition to being as formal as needed for the case if babbling around leaves room for misconception. We hope what we have written is explicit and comprehensible so that you can quickly read it without puzzling over what we have written too much. The prize we paid for this strategy is the large amount of space for comparatively little content. Also note that we are usually way more formal in chapter \ref{chap:cattg} than in all the other chapters.
\\
Mathematical definitions are \textbf{bold} and usually presented in the style {\glqq}$\mathcal{F}$ is called \textbf{NAME OF THE FORMULA $\mathcal{F}$}{\grqq} or {\glqq}$\mathcal{F}_{1}$ is \textbf{NAME OF THE FORMULA $\mathcal{F}_{1}$} if $\mathcal{F}_{2}${\grqq} where the tokens $\mathcal{F},\mathcal{F}_{1},\mathcal{F}_{2}$ are variables for formulas. In the second case $\mathcal{F}_{2}$ is often only semi-formal in practice. We also use parentheses in the \textbf{bold} text to indicate which part of the name we allow to be implicit. So if we refer to a definition then it suffices to say the part not in parentheses (and perhaps parts of the parenthesized name) if we do not have to fear misunderstandings. Mathematical statements are made and proved in the running text. But if we later want to refer to it or if the statement is of particular importance we use numbered theorem environments followed by (unnumbered) proof environments which terminate with a $\square$. We think this is best for an ideal mix of comprehension and readability. Most important proofs in ordinary category theory are done in a quite explicit fashion. In other places we try to give references with sufficient proof of the statements of discourse. But we frequently leave proofs as reader's exercise if they are not so important or easy or tedious and not enlightening. And sometimes we just sketch a proof (without a convincing reason). A last word on examples. They are treated in the same fashion as (long) statements (since they usually are). However, we want to emphasize that they are rarely for illustration purposes only in these notes so don't just skip them carelessly if you think you have already understood the aspect they illustrate.
