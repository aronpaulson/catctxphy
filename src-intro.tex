%\nocite{c55c71e8}
%\nocite{e837ef86}
%\nocite{a565d200}
%\nocite{dc6f686f}
%\nocite{1ba1603e}
%\nocite{52fbba46}
%\nocite{8b5861fc}
%\nocite{wiki-nlab0000}
%\nocite{wiki-pedia0en}
These notes are intended as an introductory text on category theory in the context of physics. The idea of the notes is to serve as introductory text on category theory and how to apply it to modern physics. For that purpose we develop category along physics in the main part of these notes (chapter \ref{chap:initcontext}) surrounded by chapters with physical context. Chapter \ref{chap:initcontext} preceding the category part tries to motivate the use of category theory in physics while chapter \ref{chap:termcontext} succeeding the category part discusses what to do with the gained knowledge eventually leading to an account of physics \cite{a565d200} leveraging higher category theory.
\\
For reasons of modularity we refrain from writing a more thorough introduction defering it to the other chapters but only talk about the mathematical requirements to understand the notes and some style things in the rest of this chapter
\\\\
Actually, the bar is low to read these notes. While the ideal reader has a solid mathematical background (let's just say undergraduate level, at least) we claim that a good part of these notes is understandable with way less knowledge on mathematics. This is because category theory can be regarded as being at the very beginning of mathematics. So the target group is anybody with some basic mathematical knowledge (we will say {\glqq}what{\grqq} in a moment) possibly with an interest in physics.
\\
Now what do we require you to know? Well, there are several levels on which one can read these notes:
\begin{enumerate}
\item[(1)]
If you want to understand the category theory part only, then there is not much you are required to know already. It suffices to be familiar with very basic mathematics, which we claim someone doing mathematics on university for one year or so is:
\begin{enumerate}
\item[$\bullet$]
Reasoning in classical logic about mathematical objects. This in particular includes an understanding of first-order theories on an intuitive level. That is, you should be able to read and understand formulas such as $\exists x \forall y(y \in x \Rightarrow \ldots)$ and you should have a feeling for what a mathematical proof is.
\item[$\bullet$]
Set theory and some fundamental constructions in set theory. You are not required to know formal set theory (such as ZFC) but only what one can do with sets in ordinary mathematics on an intuitive level. In particular, you should know about unions, intersections, ordered pairs and perhaps some other fundamental stuff we forgot to list here.
\item[$\bullet$]
Relations and particularly equivalence relations.
\item[$\bullet$]
Functions. Here we mean functions between sets. At this point, already note that we sometimes differ functions from {\glqq}assignments{\grqq}\footnote{this is a non-standard term (sometimes people speak of definable functions in this context)} terminologically. In our terminology assignments not only map the elements of a set - an object of the mathemtical universe - to elements of some further set, but (a part of) the objects in the mathematical universe to objects of the mathematical universe.
\item[$\bullet$]
Ordered sets. We require knowledge about preordered sets, partially ordered sets, totally ordered sets and well-ordered sets.
\end{enumerate}
This is actually enough for most of the notes. Yet, it is certainly better to know more about mathematics for a deeper understanding. For example, an understanding of what mathematics is (there are different ideas on that) is certainly helpful though it is not easy to get and especially when beginning to learn about it it might raise more questions than it answers. However, you can take action on another level to improve the understanding of these notes. Many of our examples originate in certain traditional fields of mathematics: algebra and topology. A proper understanding of these fields will help you understand the examples and thus category theory better. We will discuss most things from algebra and topology we need for reasons of self-containment. But do not expect a too pedagogical discussion. So it would be good to already know about algebra, topology and perhaps a bit on the foundations of mathematics.
\item[(2)]
If you want to understand the whole category theory part then there is no getting round homotopy theory. Homotopy theory is deeply linked to category theory and higher category theory. This is one of the major aspects of these notes as you will very soon realize. Also, cohomology will play a big role in these notes. We do not want to spoil too much here so we only say a perfectly prepared reader is already familiar with some traditional algebraic topology.
\item[(3)]
If you also want to understand the context then you should know a bit about basic physics and quite a bit about differential geometry.
\end{enumerate}
Note that we sometimes write terms in \textit{italics} when the reader is not expected to already know these. Moreover we also want to explicitly emphasize that you should just skip things you do not understand. You should then still be able to see what is going on. At least, this is our intention.
\\
Last but not least note that we propose references along the notes on some of the stuff we discussed as requirements. But there are a few sources the content of these notes is based on we want to highlight explicitly. The obvious one is wikipedia \cite{wiki-pedia0en}. Probably anybody uses this as general purpose source and so do we. But for (higher) category theory and many things around this there is another such thing: the nLab \cite{wiki-nlab0000}. If you are seriously interested in what we do here then this is your friend. The category theory we do is based on that and on the following three sources:
\begin{enumerate}
\item[$\bullet$]
Mac Lane's traditional book \cite{e837ef86}
\item[$\bullet$]
Leinster's introductory book \cite{dc6f686f}
\item[$\bullet$]
Riehl's modern book \cite{52fbba46}
\end{enumerate}
The {\glqq}wisdom{\grqq} in these notes is arguably due to Mac Lane's and Moerdijk's book on sheaves and topoi \cite{c55c71e8}. Last we should not forget about homotopy theory:
\begin{enumerate}
\item[$\bullet$]
Hatcher's well known algebraic topology book \cite{8b5861fc} for a traditional view
\item[$\bullet$]
The Univalent Foundations Program book on homotopy type theory \cite{1ba1603e} for a modern synthetic approach
\end{enumerate}
Most of these sources are heavily cited in these notes.
\\\\
Now we have to talk about the style of this text a bit. Our focus is on intuition but also on being precise as far as possible. We tried to motivate anything on an intuitive level on the one hand while additionally be as formal as needed on the other. This is because being fully formal is unambiguous but hard to understand for humans. So we think it is better to babble much around things in a natural language in addition to being as formal as needed for the case if babbling around leaves room for misconception. We hope what we have written is explicit and comprehensible so that you can quickly read it without puzzling over what we have written too much. The price we paid for this strategy is the large amount of space for comparatively little content. Also note that we are usually way more formal in chapter \ref{chap:cattg} than in all the other chapters.
\\
Mathematical definitions are \textbf{bold} and usually presented in the style {\glqq}$\mathcal{F}$ is called \textbf{NAME OF THE FORMULA $\mathcal{F}$}{\grqq} or {\glqq}$\mathcal{F}_{1}$ is \textbf{NAME OF THE FORMULA $\mathcal{F}_{1}$} if $\mathcal{F}_{2}${\grqq} where the tokens $\mathcal{F},\mathcal{F}_{1},\mathcal{F}_{2}$ are variables for formulas. In the second case $\mathcal{F}_{2}$ is often only semi-formal in practice. We also use parentheses in the \textbf{bold} text to indicate which part of the name we allow to be implicit. So if we refer to a definition then it suffices to say the part not in parentheses (and perhaps parts of the parenthesized name) if we do not have to fear misunderstandings. Mathematical statements are made and proved in the running text. But if we later want to refer to it or if the statement is of particular importance we use numbered theorem environments followed by (unnumbered) proof environments which terminate with a $\square$. We think this is best for an ideal mix of comprehension and readability. Most important proofs in ordinary category theory are done in a quite explicit fashion. In other places we try to give references with sufficient proof of the statements of discourse. But we frequently leave proofs as reader's exercise if they are not so important or easy or tedious and not enlightening. And sometimes we just sketch a proof (without any convincing reason). A last word on examples: they are treated in the same fashion as (long) statements (since they usually are). However, we want to emphasize that they are rarely for illustration purposes only in these notes so don't just skip them carelessly if you think you have already understood the aspect they illustrate.
